% splash_intro.tex
%
% 2014-10-29 -- created
% 2015-08-23 -- last updated
%
% ~~~~~~~~~
% citation:
% ~~~~~~~~~
% T. W. Davis, I. C. Prentice, B. D. Stocker, R. J. Whitley, H. Wang, B. J.
% Evans, A. V. Gallego-Sala, M. T. Sykes, and W. Cramer, Simple process-led
% algorithms for simulating habitats (SPLASH): Modelling radiation evapo-
% transpiration and plant-available moisture, Geoscientific Model Development, 
% 2015 (in progress)
%
% ~~~~~~~~~~~~
% description:
% ~~~~~~~~~~~~
% This TEX file contains the first chapter of the SPLASH code book.
%
% ~~~~~~~~~~
% changelog:
% ~~~~~~~~~~
% 01. modularized chapter [14.10.29]
% 02. newline for each sentence [14.10.29]
% --> simpler for Git version control
% 03. updated values in Table [14.11.25]
% 04. updated variable names [14.11.25]
% 05. updated intro & constant references [15.01.19]
% 06. renaming (addresses issue #3) [15.08.23]
%
%% \\\\\\\\\\\\\\\\\\\\\\\\\\\\\\\\\\\\\\\\\\\\\\\\\\\\\\\\\\\\\\\\\\\\\\\\ %%
%% PART 1 -- INTRODUCTION
%% //////////////////////////////////////////////////////////////////////// %%
\section{Introduction}
\label{sec:intro}
With a growing need of global ecophysiological datasets for the study of vegetation dynamics under changing climate scenarios, the simple process-led algorithms for simulating habitats (SPLASH) present an analytical solution that provides daily, monthly, and annual estimates of key modeling parameters (e.g., net radiation, actual evapotranspiration, soil moisture index).

%% \\\\\\\\\\\\\\\\\\\\\\\\\\\\\\\\\\\\\\\\\\\\\\\\\\\\\\\\\\\\\\\\\\\\\\\\ %%
%% PART 1.2 -- THEORY
%% //////////////////////////////////////////////////////////////////////// %%
\subsection{Theory}
\label{sec:theory}
The methodology is based on the steps outlined in \parencite{cramer88}:

\begin{enumerate}
	\item Daily
	\begin{enumerate}
		\item Estimate the evaporative supply rate, $S_w$ (\S \ref{sec:sw})
		\item Calculate the heliocentric longitudes, 
		      $\nu$ and $\lambda$ (\S \ref{sec:lambda})
		\item Calculate the distance factor, $d_r$ 
		      (\S \ref{sec:dr})
		\item Calculate the declination angle, $\delta$ 
		      (\S \ref{sec:delta})
		\item Calculate the sunset angle, $h_s$ (Eq. \ref{eq:hs})
		\item Calculate daily extraterrestrial solar radiation flux, $H_o$ 
		      (\S \ref{sec:dra})
		\item Estimate transmittivity, $\tau$ (\S \ref{sec:rs})
		\item Calculate daily photosynthetic photon flux density, $Q_n$ 
		      (\S \ref{sec:dppfd})
		\item Estimate net longwave radiation flux, $I_{LW}$ (\S \ref{sec:rn})
		\item Calculate net radiation cross-over hour angle, $h_n$ 
		      (Eq. \ref{eq:hn})
		\item Calculate daytime net radiation, $H_{N}$ (\S \ref{sec:drn})
		\item Calculate nighttime net radiation, $H_{N}^{\ast}$ (\S \ref{sec:drnn})
		\item Calculate energy conversion factor, $E_{con}$ (\S \ref{sec:econ})
		\item Estimate daily condensation, $C_n$ (\S \ref{sec:cond})
		\item Estimate daily equilibrium evapotranspiration, $E_n^q$ 
		      (\S \ref{sec:deet})
		\item Estimate daily potential evapotranspiration, $E_n^p$ 
		      (\S \ref{sec:dpet})
		\item Calculate the intersection hour angle, $h_i$ (Eq. \ref{eq:hi})
		\item Estimate daily actual evapotranspiration, $E_n^a$ 
		      (\S \ref{sec:daet})
		\item Update daily soil moisture, $W_n$ (\S \ref{sec:dw})
	\end{enumerate}
	\item Monthly
	\begin{enumerate}
		\item Sum monthly totals of $E_m^a$, $E_m^p$, $E_m^q$ and $Q_m$
		\item Calculate monthly Cramer-Prentice moisture index, $\alpha$ 
		      (\S \ref{sec:alpha})
		\item Calculate monthly climatic water deficit, $\Delta E_m$ 
		      (\S \ref{sec:cwd})
	\end{enumerate}
	\item Yearly
	\begin{enumerate}
		\item Test whether soil on 31 December agrees with initial conditions
	\end{enumerate}
\end{enumerate}

%% \\\\\\\\\\\\\\\\\\\\\\\\\\\\\\\\\\\\\\\\\\\\\\\\\\\\\\\\\\\\\\\\\\\\\\\\ %%
%% PART 1.2 -- KEY OUTPUTS
%%///////////////////////////////////////////////////////////////////////// %%
\subsection{Key Outputs}
\label{sec:outputs}
The key daily outputs from this model are:
\begin{enumerate}
	\item extraterrestrial solar irradiation ($H_o$), J m$^{-2}$
	\item net radiation ($H_N$), J m$^{-2}$
	\item photosynthetic photon flux density ($Q_n$), mol $m^{-2}$
	\item condensation ($C_n$), mm
	\item soil moisture ($W_n$), mm
	\item runoff ($RO$), mm
	\item equilibrium evapotranspiration ($E_n^q$), mm
	\item potential evapotranspiration ($E_n^p$), mm
	\item actual evapotranspiration ($E_n^a$), mm
\end{enumerate}

The key monthly outputs from this model are:
\begin{enumerate}
	\item PPFD ($Q_m$), mol$\cdot$m$^{-2}$
	\item equilibrium evapotranspiration ($E_m^q$), mm
	\item potential evapotranspiration ($E_m^p$), mm
	\item Cramer-Prentice moisture index ($\alpha$), unitless
	\item climatic water deficit ($\Delta E_m$), mm
\end{enumerate}


%% \\\\\\\\\\\\\\\\\\\\\\\\\\\\\\\\\\\\\\\\\\\\\\\\\\\\\\\\\\\\\\\\\\\\\\\\ %%
%% PART 1.3 -- MODEL INPUTS
%%///////////////////////////////////////////////////////////////////////// %%
\subsection{Model Inputs}
\label{sec:inputs}
The model simulation of radiation fluxes requires basic inputs on the time of the year (i.e., year, month, and day) and geographic position:

\begin{enumerate}
	\item longitude ($\theta_{lon}$), degrees---for subdaily results only
	\item latitude ($\phi$), degrees
	\item elevation ($z$), meters
\end{enumerate} 
 
For modeling evaporation, the basic daily meteorological variables needed are: 
\begin{enumerate}
	\item air temperature ($T_{air}$, $^{\circ}$C
	\item precipitation ($P_n$), mm
	\item fraction of sunlight hours ($S_f$), percent
\end{enumerate}

In most cases, daily values of the necessary meteorological variables are not available (especially for global coverage). 
It is possible, then, to use CRU TS datasets\footnotemark, which provide 0.5$^{\circ}$ resolution global monthly climate variables including: monthly mean daily air temperature, $^{\circ}$C (TMP), monthly precipitation totals, millimeters (PRE), and percent cloudiness, unitless (CLD).\footnotetext{\url{http://badc.nerc.ac.uk/view/badc.nerc.ac.uk\textunderscore \textunderscore ATOM\textunderscore \textunderscore dataent\textunderscore 1256223773328276}} 

Based on CRU TS climatic data, the mean daily temperature, $T_{air}$, may be assumed constant for each day in the month. 
The daily precipitation, $P_n$, may be assumed as a constant fraction of the monthly precipitation (i.e., the monthly total precipitation divided by the number of days in the month, $N_m$). 
The percent cloudiness data is derived on fractional sunshine hours \parencite{harris14}; therefore, the fraction of sunshine hours, $S_f$, may be calculated (albeit only loosely analogous) as the complementary fraction of cloudiness (i.e., $1-$CLD) and may be assumed constant for each day in the month.

%% \\\\\\\\\\\\\\\\\\\\\\\\\\\\\\\\\\\\\\\\\\\\\\\\\\\\\\\\\\\\\\\\\\\\\\\\ %%
%% PART 1.4 -- MODEL CONSTANTS
%%///////////////////////////////////////////////////////////////////////// %%
\subsection{Model Constants}
\label{sec:constants}
Table \ref{tab:constants} presents the constant values used in this model with the corresponding symbol used in this document and variable name used in the coded environment. 

%% ------------------------------------------------------------------------ %%
%% tab:constants | Constants used in the SPLASH model
%% ------------------------------------------------------------------------ %%
\nomenclature{$A$}{Empirical constant for net radiation flux}%
\nomenclature{$a$}{Length of the semi-major axis, km}%
\nomenclature{$\beta_{sw}$}{Shortwave albedo, unitless}%
\nomenclature{$\beta_{vis}$}{PAR albedo, unitless}%
\nomenclature{$b$}{Empirical constant for net radiation flux, unitless}%
\nomenclature{$c$}{Minimum transmittivity for cloudy skies, unitless}%
\nomenclature{$S_c$}{Maximum rate of evaporation, mm$\cdot$h$^{-1}$}%
\nomenclature{$d$}{Angular coefficient of transmittivity, unitless}%
\nomenclature{$e$}{Earth's orbital eccentricity, unitless}%
\nomenclature{$\epsilon$}{Obliquity of the elliptic, degrees}%
\nomenclature{$\text{fFEC}$}{From flux to energy conversion, $\mu$mol$\cdot$J$^{-1}$}%
\nomenclature{$g$}{Gravitational acceleration, m$\cdot$s$^{-2}$}%
\nomenclature{$GM$}{Standard gravity of the sun, km$^{3}\cdot$s$^{-2}$}%
\nomenclature{$L$}{Temperature lapse rate, K$\cdot$m$^{-1}$}%
\nomenclature{$M_a$}{Molecular weight of dry air, kg$\cdot$mol$^{-1}$}%
\nomenclature{$M_v$}{Molecular weight of water vapor, kg$\cdot$mol$^{-1}$}%
\nomenclature{$P_o$}{Base pressure, Pa}%
\nomenclature{$R$}{Universal gas constant, J$\cdot$mol$^{-1}\cdot$K$^{-1}$}%
\nomenclature{$\sigma_{sb}$}{Stefan-Boltzmann constant, W$\cdot$m$^{-2}\cdot$K$^{-4}$}%
\nomenclature{$T_o$}{Base temperature, K}%
\nomenclature{$\omega$}{Entrainment factor, unitless}%
\nomenclature{$\tilde{\omega}$}{Longitude of the perihelion, degrees}
\begin{longtable}{c c r p{6cm}}
	\caption{Constants used in the SPLASH model. \label{tab:constants}} \\
	\hline 
	\textbf{Symbol} & \textbf{Variable} & \textbf{Value} & 
	\textbf{Definition} \\
    \hline
    \endfirsthead
    \caption{Constants used in the STASH model (continued).} \\
	\hline 
	\textbf{Symbol} & \textbf{Variable} & \textbf{Value} & 
	\textbf{Definition} \\
    \hline
	\endhead
    $A$ & \texttt{kA} & 107 & 
    	Constant for $R_{nl}$, W$\cdot$m$^{-2}$ 
        \parencite{monteith90} \\ 
    $a$ & \texttt{ka} & 1.49598 &
        Semi-major axis, $\times 10^8$ km \parencite{allen73} \\
    $\beta_{sw}$ & \texttt{kalb\textunderscore sw} & 0.17 &
        Shortwave albedo, unitless 
        \parencite{federer68} \\
    $\beta_{vis}$ & \texttt{kalb\textunderscore vis} & 0.03 &
        PAR albedo, unitless 
        \parencite{sellers85} \\
    $b$ & \texttt{kb} & 0.20 &
        Constant for $R_{nl}$, unitless  
        \parencite{linacre68} \\
    $c$ & \texttt{kc} & 0.25 & 
        Cloudy transmittivity, unitless 
        \parencite{linacre68} \\
    $d$ & \texttt{kd} & 0.50 & 
        Angular coefficient of transmittivity, unitless 
        \parencite{linacre68} \\
    $e$ & \texttt{ke} & 0.01670 & 
        Eccentricity (2000 CE), unitless 
        \parencite{berger78} \\
    $\epsilon$ & \texttt{keps} & 23.44 & 
        Obliquity (2000 CE), degrees 
        \parencite{berger78} \\
    fFEC & \texttt{kfFEC} & 2.04 & 
        From flux to energy, $\mu$mol$\cdot$J$^{-1}$  
        \parencite{meek84} \\
    $g$ & \texttt{kG} & 9.80665 &
        Gravitational acceleration, m$\cdot$s$^{-2}$
        \parencite{allen73} \\
    $GM$ & \texttt{kGM} & 1.32712 &
        Standard gravity, $\times 10^{11}$ km$^3\cdot$s$^{-2}$ \\
    $I_{sc}$ & \texttt{kGsc} & 1360.8 &
        Solar constant, W$\cdot$m$^{-2}$ 
        \parencite{kopp11} \\
    $L$ & \texttt{kL} & 0.0065 & 
        Lapse rate, K$\cdot$m$^{-1}$ 
        \parencite{allen73} \\
    $M_a$ & \texttt{kMa} & 0.028963 &
        Molecular weight of dry air, kg$\cdot$mol$^{-1}$ 
        \parencite{tsilingiris08} \\
    $M_v$ & \texttt{kMv} & 0.01802 &
        Molecular weight of water vapor, kg$\cdot$mol$^{-1}$
        \parencite{tsilingiris08} \\
	$\omega$ & \texttt{kw} & 0.26 & 
        Entrainment, unitless 
        \parencite{lhomme97,priestley72} \\    
    $\tilde{\omega}$ & \texttt{komega} & 283 & 
        Longitude of perihelion (2000 CE), degrees 
        \parencite{berger78} \\
    $P_o$ & \texttt{kPo} & 101325 &
        Base pressure, Pa 
        \parencite{allen73} \\
    $R$ & \texttt{kR} & 8.31447 &
        Universal gas constant, J$\cdot$mol$^{-1}\cdot$K$^{-1}$ 
        \parencite{moldover88} \\
    $\sigma_{sb}$ & \texttt{ksb} & 5.670373 &
        Stefan-Boltzmann constant, $\times 10^{-8}$ 
        W$\cdot$m$^{-2}\cdot$K$^{-4}$ \\
    $S_c$ & \texttt{kCw} & 1.05 & 
        Supply constant, mm$\cdot$hr$^{-1}$  
        \parencite{federer82} \\
    $T_o$ & \texttt{kTo} & 288.15 &
        Base temperature, K 
        \parencite{berberansantos97} \\
    $W_m$ & \texttt{kWm} & 150 &
        Soil moisture capacity, mm  
        \parencite{cramer88} \\
        \hline
\end{longtable}

% @TODO: model variables table