% stash_intro.tex
%
% written by Tyler W. Davis
% Imperial College London
%
% 2014-10-29 -- created
% 2014-10-29 -- last updated
%
% ------------
% description:
% ------------
% This TEX file contains the first chapter of the STASH 2.0 code book.
%
% ----------
% changelog:
% ----------
% 01. modularized chapter [14.10.29]
% 02. newline for each sentence [14.10.29]
% --> simpler for Git version control
%
%% \\\\\\\\\\\\\\\\\\\\\\\\\\\\\\\\\\\\\\\\\\\\\\\\\\\\\\\\\\\\\\\\\\\\\\\\ %%
%% PART 1 -- INTRODUCTION
%% //////////////////////////////////////////////////////////////////////// %%
\section{Introduction}
\label{sec:intro}
This work aims to model monthly global radiation, evaporation, and soil moisture quantities and indexes using simple but theoretically-based simulation.

%% \\\\\\\\\\\\\\\\\\\\\\\\\\\\\\\\\\\\\\\\\\\\\\\\\\\\\\\\\\\\\\\\\\\\\\\\ %%
%% PART 1.2 -- THEORY
%% //////////////////////////////////////////////////////////////////////// %%
\subsection{Theory}
\label{sec:theory}
The methodology is based on the psuedo-code presented by \parencite{cramer88}:

\begin{enumerate}
	\item Daily
	\begin{enumerate}
		\item Estimate the evaporative supply rate, $S_w$ (\S \ref{sec:sw})
		\item Calculate (or estimate) the heliocentric longitudes, 
		      $\nu$ and $\lambda$ (\S \ref{sec:lambda})
		\item Calculate (or estimate) the distance factor, $d_r$ 
		      (\S \ref{sec:dr})
		\item Calculate (or estimate) the declination angle, $\delta$ 
		      (\S \ref{sec:delta})
		\item Calculate the sunset angle, $H_s$ (Eq. \ref{eq:hs})
		\item Calculate daily extraterrestrial solar radiation flux, $R_{Da}$ 
		      (\S \ref{sec:dra})
		\item Estimate transmittivity, $\tau$ (\S \ref{sec:rs})
		\item Calculate daily photosynthetic photon flux density, PPFD$_D$ 
		      (\S \ref{sec:dppfd})
		\item Estimate net longwave radiation flux, $R_{nl}$ (\S \ref{sec:rn})
		\item Calculate net radiation cross-over hour angle, $H_n$ 
		      (Eq. \ref{eq:hn})
		\item Calculate daytime net radiation, $R_{Dn}$ (\S \ref{sec:drn})
		\item Calculate nighttime net radiation, $R_{Dnn}$ (\S \ref{sec:drnn})
		\item Calculate energy conversion factor, $E_{con}$ (\S \ref{sec:econ})
		\item Estimate daily condensation, $W_c$ (\S \ref{sec:cond})
		\item Estimate daily equilibrium evapotranspiration, EET$_D$ 
		      (\S \ref{sec:deet})
		\item Estimate daily potential evapotranspiration, PET$_D$ 
		      (\S \ref{sec:dpet})
		\item Calculate the intersection hour angle, $H_i$ (Eq. \ref{eq:hi})
		\item Estimate daily actual evapotranspiration, AET$_D$ 
		      (\S \ref{sec:daet})
		\item Update daily soil moisture, $W_n$ (\S \ref{sec:dw})
	\end{enumerate}
	\item Monthly
	\begin{enumerate}
		\item Sum monthly totals of AET$_D$, PET$_D$, EET$_D$ and PPFD$_D$
		\item Calculate monthly Cramer-Prentice moisture index, $\alpha$ 
		      (\S \ref{sec:alpha})
		\item Calculate monthly climatic water deficit, $\Delta$E 
		      (\S \ref{sec:cwd})
	\end{enumerate}
	\item Yearly
	\begin{enumerate}
		\item Test whether soil on 31 December agrees with initial conditions
	\end{enumerate}
\end{enumerate}

%% \\\\\\\\\\\\\\\\\\\\\\\\\\\\\\\\\\\\\\\\\\\\\\\\\\\\\\\\\\\\\\\\\\\\\\\\ %%
%% PART 1.2 -- KEY OUTPUTS
%%///////////////////////////////////////////////////////////////////////// %%
\subsection{Key Outputs}
\label{sec:outputs}
The key outputs from this model are:
\begin{enumerate}
	\item monthly PPFD, mol$\cdot$m$^{-2}$
	\item monthly equilibrium evapotranspiration (EET), mm
	\item monthly potential evapotranspiration (PET), mm
	\item monthly Cramer-Prentice moisture index ($\alpha$), unitless
	\item monthly climatic water deficit ($\Delta$E), mm
\end{enumerate}

%% \\\\\\\\\\\\\\\\\\\\\\\\\\\\\\\\\\\\\\\\\\\\\\\\\\\\\\\\\\\\\\\\\\\\\\\\ %%
%% PART 1.3 -- MODEL INPUTS
%%///////////////////////////////////////////////////////////////////////// %%
\subsection{Model Inputs}
\label{sec:inputs}
The model simulation of radiation fluxes requires basic inputs on the time of the year (i.e., year, month, and day) and geographic position (i.e., longitude, $\theta_{lon}$, latitude, $\phi$, and elevation, $z$). 
For modeling evaporation, the basic meteorological variables needed are: air temperature, precipitation, and fraction of sunlight hours. 

In most cases, daily values of the necessary meteorological variables are not available (especially for global coverage). 
It is possible, then, to use CRU TS datasets\footnotemark, which provide 0.5$^{\circ}$ resolution global monthly climate variables including: monthly mean daily air temperature, $^{\circ}$C (TMP), monthly precipitation totals, millimeters (PRE), and percent cloudiness, unitless (CLD).\footnotetext{\url{http://badc.nerc.ac.uk/view/badc.nerc.ac.uk\textunderscore \textunderscore ATOM\textunderscore \textunderscore dataent\textunderscore 1256223773328276}} 

Based on CRU TS climatic data, the mean daily temperature, $T_c$, may be assumed constant for each day in the month. 
The daily precipitation, $P_n$, may be assumed as a constant fraction of the monthly precipitation (i.e., the monthly total precipitation divided by the number of days in the month, $N_m$). 
The percent cloudiness data is derived on fractional sunshine hours \parencite{harris14}; therefore, the fraction of sunshine hours, $S_f$, may be calculated (albeit only loosely analogous) as the complementary fraction of cloudiness (i.e., $1-$CLD) and may be assumed constant for each day in the month.

%% \\\\\\\\\\\\\\\\\\\\\\\\\\\\\\\\\\\\\\\\\\\\\\\\\\\\\\\\\\\\\\\\\\\\\\\\ %%
%% PART 1.4 -- MODEL CONSTANTS
%%///////////////////////////////////////////////////////////////////////// %%
\subsection{Model Constants}
\label{sec:constants}
Table \ref{tab:constants} presents the constant values used in this model with the corresponding symbol used in this document and variable name used in the coded environment. 

%% ------------------------------------------------------------------------ %%
%% tab:constants | Constants used in STASH model
%% ------------------------------------------------------------------------ %%
\nomenclature{$A$}{Empirical constant for net radiation flux}%
\nomenclature{$a$}{Length of the semi-major axis, km}%
\nomenclature{$\beta_{sw}$}{Shortwave albedo, unitless}%
\nomenclature{$\beta_{vis}$}{PAR albedo, unitless}%
\nomenclature{$b$}{Empirical constant for net radiation flux, unitless}%
\nomenclature{$c$}{Minimum transmittivity for cloudy skies, unitless}%
\nomenclature{$C_w$}{Maximum rate of evaporation, mm$\cdot$hr$^{-1}$}%
\nomenclature{$d$}{Angular coefficient of transmittivity, unitless}%
\nomenclature{$e$}{Earth's orbital eccentricity, unitless}%
\nomenclature{$\epsilon$}{Obliquity of the elliptic, degrees}%
\nomenclature{$\text{fFEC}$}{From flux to energy conversion, $\mu$mol$\cdot$J$^{-1}$}%
\nomenclature{$g$}{Gravitational acceleration, m$\cdot$s$^{-2}$}%
\nomenclature{$GM$}{Standard gravity of the sun, km$^{3}\cdot$s$^{-2}$}%
\nomenclature{$L$}{Temperature lapse rate, K$\cdot$m$^{-1}$}%
\nomenclature{$M_a$}{Molecular weight of dry air, kg$\cdot$mol$^{-1}$}%
\nomenclature{$M_v$}{Molecular weight of water vapor, kg$\cdot$mol$^{-1}$}%
\nomenclature{$P_o$}{Base pressure, Pa}%
\nomenclature{$R$}{Universal gas constant, J$\cdot$mol$^{-1}\cdot$K$^{-1}$}%
\nomenclature{$\sigma_{sb}$}{Stefan-Boltzmann constant, W$\cdot$m$^{-2}\cdot$K$^{-4}$}%
\nomenclature{$T_o$}{Base temperature, K}%
\nomenclature{$w$}{Entrainment factor, unitless}%
\nomenclature{$\omega$}{Longitude of the perihelion, degrees}
\begin{longtable}{c c r p{6cm}}
	\caption{Constants used in the STASH model. \label{tab:constants}} \\
	\hline 
	\textbf{Symbol} & \textbf{Variable} & \textbf{Value} & 
	\textbf{Definition} \\
    \hline
    \endfirsthead
    \caption{Constants used in the STASH model (continued).} \\
	\hline 
	\textbf{Symbol} & \textbf{Variable} & \textbf{Value} & 
	\textbf{Definition} \\
    \hline
	\endhead
    $A$ & \texttt{kA} & 107 & 
    	Constant for $R_{nl}$, W$\cdot$m$^{-2}$ 
        \parencite{monteith90} \\ 
    $a$ & \texttt{ka} & 1.496 &
        Semi-major axis, $\times 10^8$ km \\
    $\beta_{sw}$ & \texttt{kalb\textunderscore sw} & 0.17 &
        Shortwave albedo, unitless 
        \parencite{federer68} \\
    $\beta_{vis}$ & \texttt{kalb\textunderscore vis} & 0.03 &
        PAR albedo, unitless 
        \parencite{sellers85} \\
    $b$ & \texttt{kb} & 0.20 &
        Constant for $R_{nl}$, unitless  
        \parencite{linacre68} \\
    $c$ & \texttt{kc} & 0.25 & 
        Cloudy transmittivity, unitless 
        \parencite{linacre68} \\
    $C_w$ & \texttt{kCw} & 1.05 & 
        Supply constant, mm$\cdot$hr$^{-1}$  
        \parencite{federer82} \\
    $d$ & \texttt{kd} & 0.50 & 
        Angular coefficient of transmittivity, unitless 
        \parencite{linacre68} \\
    $e$ & \texttt{ke} & 0.01670 & 
        Eccentricity (2000 CE), unitless 
        \parencite{berger78} \\
    $\epsilon$ & \texttt{keps} & 23.44 & 
        Obliquity (2000 CE), degrees 
        \parencite{berger78} \\
    fFEC & \texttt{kfFEC} & 2.04 & 
        From flux to energy, $\mu$mol$\cdot$J$^{-1}$  
        \parencite{meek84} \\
    $g$ & \texttt{kG} & 9.80665 &
        Gravitational acceleration, m$\cdot$s$^{-2}$ \\
    $GM$ & \texttt{kGM} & 1.32712 &
        Standard gravity, $\times 10^{11}$ km$^3\cdot$s$^{-2}$ \\
    $G_{sc}$ & \texttt{kGsc} & 1360.8 &
        Solar constant, W$\cdot$m$^{-2}$ 
        \parencite{kopp11} \\
    $L$ & \texttt{kL} & 0.0065 & 
        Lapse rate, K$\cdot$m$^{-1}$ 
        \parencite{cavcar00} \\
    $M_a$ & \texttt{kMa} & 0.028963 &
        Molecular weight of dry air, kg$\cdot$mol$^{-1}$ 
        \parencite{tsilingiris08} \\
    $M_v$ & \texttt{kMv} & 0.01802 &
        Molecular weight of water vapor, kg$\cdot$mol$^{-1}$
        \parencite{tsilingiris08} \\
    $P_o$ & \texttt{kPo} & 101325 &
        Base pressure, Pa 
        \parencite{cavcar00} \\
    $R$ & \texttt{kR} & 8.314 &
        Universal gas constant, J$\cdot$mol$^{-1}\cdot$K$^{-1}$ \\
    $\sigma_{sb}$ & \texttt{ksb} & 5.670373 &
        Stefan-Boltzmann constant, $\times 10^{-8}$ 
        W$\cdot$m$^{-2}\cdot$K$^{-4}$ \\
    $T_o$ & \texttt{kTo} & 298.15 &
        Base temperature, K 
        (Prentice, unpublished) \\
    $W_m$ & \texttt{kWm} & 150 &
        Soil moisture capacity, mm  
        \parencite{cramer88} \\
    $w$ & \texttt{kw} & 0.26 & 
        Entrainment, unitless 
        \parencite{lhomme97,priestley72} \\
    $\omega$ & \texttt{komega} & 283 & 
        Longitude of perihelion (2000 CE), degrees 
        \parencite{berger78} \\
        \hline
\end{longtable}

% @TODO: model variables table